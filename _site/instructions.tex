\documentclass[12pt,legalpaper]{article}
\author{Ray Huber}
\title{Spark 3D Printer Basic Guide}

\usepackage{graphicx}
\graphicspath{{./}}

\begin{document}

\maketitle


\pagebreak

\section{Basics}
\paragraph{Orientation.}
This is the basic file system viewer in Ubuntu and the folders on this machine for printer use are marked.
\begin{enumerate}
    \item STL files (*.stl) go in the models folder.
    \item Projects in OrcaSlicer can be saved in the projects folder.
    \item Anything downloaded in Firefox automagically ends up in the Downloads folder.
\end{enumerate}
\includegraphics*{file system.jpg}

When you launch OrcaSlicer this is the basic screen you get.

\includegraphics*{home screen.jpg}

From here you can open an existing project or start a new one (these options are also in the dropdown menu in the upper left.)

\paragraph{Putting objects in a project.}

This is the basic layout you see when a new project is started.

\includegraphics*{empty.jpg}

Objects can be added by dragging and dropping from the file system viewer. Once on the plate objects can be adjusted
using the controls at the top of the window.

\includegraphics*{object contorls.jpg}

The tools in the object controls from left to right are:
\begin{enumerate}
    \item Translate: Move an object on the plate.
    \item Rotate: Rotate an object.
    \item Scale: Scale an object.
    \item Place on face: This lets you pick a face on the object to align to the build plate.
    \item Cut: Splits the object along a plane.
    \item Mesh Boolean: This allows taking two meshes and doing a boolean operation on them (e.g. uninon.)
    \item Support Painting: Allows you to force supports to or exclude supports from parts of the model.
    \item Seam Painting: Allows you to tell the slicer where to put the seams in the outer perimeter.
    \item Emboss: Allows you to imprint or extrude text on the object.
    \item Measure: Slightly worse than a banana for scale.
\end{enumerate}
The arrange tool is used to distribute multiple objects evenly around the plate.

\includegraphics*{arrange.jpg}

\paragraph{Preparing for Slicing.}
To setup for slicing we need to check the following things in the control panel on the left side of the window.
First we select the printer we want to use. The printers have name tags on top of them.

\includegraphics*{printer list.jpg}

Next we check that the correct type of filament is selected. This will normally be PLA.

\includegraphics*{filament picker.jpg}

From there we go through the build settings to setup for the best possible print results.

The first thing we look at is the quality tab. The main things we can adjust here are:
\begin{enumerate}
    \item Layer Height: This defines the layer height for layers numbered $>1$. $0.2mm$ is the default.
    Values as low as $1.5mm$ can be used for more detailed layer, and up to $3.0mm$ can be used for faster prints.
    \item First Layer Height: This defines the layer height for the first layer only. 
    It should be set between $100\%$ and $120\%$ the other layers' height.
    \item Seam Position: This is used to decide how the seams will be arranged on the object:
    \begin{enumerate}
        \item Aligned: keeps all the seams together on the object making a distinct line.
        \item Nearest: Puts the seam in a position to minimize head travel between layers. This will make the seam pattern semi random
        \item Back: Tries to hide view of the seams on the "back" of the object.
        \item Random: Places the seams in a random position on each layer.
    \end{enumerate}
\end{enumerate}

\includegraphics*{support options.jpg}

Next we pick the support options. If the object has overhanging or floating sections that need supports check the box for supports.
From there select the type of supports to use. With "Normal" the supports will be a basic rectilinear pattern. 
With "Tree" supports the slicer will grow organic looking structures to support the object.
With the "(auto)" option selected the slicer will automatically detect where supports are needed and add them in addition to any support painting that was done manually.
With the "(manual)" option the slicer will only put supports where the user has forced them.

\includegraphics*{other options.jpg}

Finally we go to the tab for Other. The main option to adjust here is the brim. "Auto" will work for most meshes.
If you want to force a brim select the appropriate kind for the mesh. And selecting "No-brim" will prevent a brim from being added.

\paragraph{Now we slice.}
Once we have all of our setttings checked hit the "Slice Plate" button in the upper right. The computer will think for a while then the view will look like below.

\includegraphics*{sliced.jpg}

From here we want to check the estimation of filament, and the printer to make sure the print will complete. 
From there we hit the "Print" button and select "Upload \& Print."
\textbf{Once the print starts make sure to ensure the first layer completes successfully.}

\paragraph{And now we're printing with filament.}
We can observe the state of the printer throught the device tab in OrcaSlicer.

\includegraphics*{status.jpg}



\end{document}